\documentclass[12pt, letterpaper]{article}
\usepackage[skip=0.25cm]{parskip}
\usepackage[total={6.5in, 8.75in}, top=1.2in, left=0.9in, includefoot]{geometry} 

\usepackage{graphicx} % images
\usepackage{enumitem} % managing lists
\usepackage{float} % precise placement

\usepackage{listings} % code
\usepackage{amssymb, amsmath}
\usepackage{outlines}

\newcommand{\F}{\mathbb{F}}
\newcommand{\R}{\mathbb{R}}
\newcommand{\N}{\mathbb{N}}
\newcommand{\Z}{\mathbb{Z}}
\newcommand{\0}{\textbf{0}}
\newcommand{\1}{\textbf{1}}
\renewcommand{\t}{\textbf{t}}
\renewcommand{\a}{\textbf{a}}
\renewcommand{\b}{\textbf{b}}
\renewcommand{\c}{\textbf{c}}

\title{
A Matrix Representation for the Combinatorics of Student Club Participation \\
\large Challenge Report 2 \\
Math 115A}
\author{Arteen Abrishami}
\date{}
\renewcommand{\abstractname}{Problem Statement}

\usepackage{titlesec}

\newcommand\almostnormalsize{\fontsize{11.42}{11.38}\selectfont}

\titleformat*{\section}{\normalsize\bfseries}
\titleformat*{\subsection}{\almostnormalsize\bfseries}

\allowdisplaybreaks

\linespread{1.5}
\let\tab\quad
\begin{document}
\maketitle
\begin{abstract}
A club is a \textbf{non-empty} set of students. No two clubs have the exact same set of students. That is, if $\exists$ two distinct clubs $C_{1}, C_{2} $ then $ C_{1} \ne C_{2} $. 

Let $ \exists n \in \N $ students, where $ \N = \{ 1, 2, 3, ... \} $. We label the students from 1 to $n$ in some arbitrary order. Similarly, let $\exists m \in \N $ clubs, labeled from 1 to $m$ in some arbitrary order. Consider the function:
$$ a_{ij} = \left\{
		\begin{array}{ll}
			1 & \text{ if club } C_i \text{ contains student } S_j \\
			0 & \text{ else} 
		\end{array}
		\right.$$

We define an \textbf{incidence matrix} $A \in M_{m \times n}(\R) = [a_{ij}] $ such that the rows of the matrix correspond to the clubs and the columns to the students.

Later, it will be important to note that addition in $\F_2$ is defined as: 
$$+_{\F_2}: \F_2 \times \F_2 \rightarrow \F_2$$ $$ a +_{\F_2} b := (a +_{\R} b)\bmod{2} $$
 for $a, b \in \F_2$. For completeness, note that multiplication in $\F_2$ is defined as: 
 $$\cdot_{\F_2}: \F_2 \times \F_2 \rightarrow \F_2$$ $$ a \cdot_{\F_2} b := (a \cdot_{\R} b)\bmod{2} $$
 for $a, b \in \F_2$. \\
 \end{abstract}

\section*{Part 1 \textmd{Counting Clubs}}

Let $n$ be the number of students and $m$ be the number of clubs that students are in.

\subsection*{(1) \textmd{rank(A)}}

The rank of the incidence matrix $A \in M_{m \times n}(\R)$ corresponds to the number of linearly independent columns in $A$. Considering that a column can be viewed as a student vector $S_{j}$, we can say that there are $rank(A)$ number of linearly independent vectors $S_{j}$, with each row representing whether they are in a given club $C_i$ or not. Equivalently, we can say that the rank corresponds to the number of students whose club participation can form (as a linear combination) the club participation of the other students (those students corresponding to a linearly dependent vector $S_{j}$).

The upper bound on the rank of $A$ is therefore $n$, as there can be at most $n$ linearly independent columns, or vectors $S_j$, spanning the column space. In other words, the upper bound on the $rank(A)$ is the number of students whose club participation is being tracked. This occurs when each student $S_j$ cannot be formed as a linear combination of the other students (or columns) in the matrix $A$.

Another way of looking at this is to say that the upper bound on $rank(A)$ is the dimension of the $Im(A)$, where $dim(Im(A)) \le n$. On the occasion that $rank(A) = n$, which we call the matrix $A$ having \emph{full rank}, we know that the club participations of every student $S_j$ are completely independent of one another, as linear combinations.

\subsection*{(2) \textmd{$AA^T$}}

Considering the matrix $AA^T$. We attempt to generate a formula for its entries in terms of the entries of $A$ and $A^T$.

\begin{align*}
A &=\begin{bmatrix}
a_{11} & ... & ... & a_{1n} \\
a_{21} & ... & ... & a_{2n} \\
... & ... & ... & ... \\
a_{m1} & ... & ... & a_{mn}
\end{bmatrix} \\
A^T &= \begin{bmatrix}
a_{11} & ... & ... & a_{m1} \\
a_{12} & ... & ... & a_{m2} \\
... & ... & ... & ... \\
a_{1n} & ... & ... & a_{mn}
\end{bmatrix} \\
AA^T &= 
\begin{bmatrix}
a_{11} & ... & ... & a_{1n} \\
a_{21} & ... & ... & a_{2n} \\
... & ... & ... & ... \\
a_{m1} & ... & ... & a_{mn}
\end{bmatrix}
\begin{bmatrix}
a_{11} & ... & ... & a_{m1} \\
a_{12} & ... & ... & a_{m2} \\
... & ... & ... & ... \\
a_{1n} & ... & ... & a_{mn}
\end{bmatrix} \\
&= 
\begin{bmatrix}
a_{11}a_{11} + a_{12}a_{12} + ... + a_{1n}a_{1n}  & ... & a_{11}a_{m1} + a_{12}a_{m2} + ... + a_{1n}a_{mn} \\
a_{21}a_{11} + a_{22}a_{12} + ... + a_{2n}a_{1n} & ... & a_{21}a_{m1} + a_{22}a_{m2} + ... + a_{2n}a_{mn}  \\
... & ... & ... \\
a_{m1}a_{11} + a_{m2}a_{12} + ... + a_{mn}a_{1n} &  ... & a_{m1}a_{m1} + a_{m2}a_{m2} + ... + a_{mn}a_{mn} 
\end{bmatrix} \\
&= 
\begin{bmatrix}
\sum_{k=1}^{n} a_{1k}a_{1k} & ... & \sum_{k=1}^{n} a_{1k}a_{mk}  \\
\sum_{k=1}^{n} a_{2k}a_{1k}  & ... & \sum_{k=1}^{n} a_{2k}a_{mk}  \\
... & ... & ... \\
\sum_{k=1}^{n} a_{mk}a_{1k} &  ... & \sum_{k=1}^{n} a_{mk}a_{mk} 
\end{bmatrix}
\end{align*}

Therefore, in general, we can say that $AA^T \in M_{m \times m}(\R) = [ a^T_{ij} ]$ where $a^T_{ij}$ is the function defined by:
$$
a^T_{ij} = \sum_{k=1}^{n} a_{ik}a_{jk}
$$

in terms of the entries in the incidence matrix $A$.

The entry $a^T_{ij}$ in $AA^T \in M_{m \times m}(\R)$, otherwise known as the \textbf{intersection matrix}, describes how many students the clubs $C_i$ and $C_j$ have in common. Note that this matrix is therefore \emph{symmetric}.

As a corrollary, the diagonal entries $\{ a^T_{11},\; a^T_{22},\; ... \}$, where $i = j$, simply denote the number of students in the the corresponding club $\{ C_{1},\; C_{2},\; ... \}$. \\

\section*{Part 2 \textmd{Two Rules on Clubs}}

To recap, a club is a \textbf{non-empty} subset of students where $C_i \ne C_j$ when $i \ne j$. That is, no two distinct clubs have the exact same set of students. 

We now impose two rules:

\tab (a) Every club must have an \emph{odd} number of members

\tab (b) Every club must have an \emph{even} number of members in common

Considering these rules, we find an upper bound on the number of possible clubs. Once again, let $n$ be the number of students and $m$ be the number of clubs that the students are in.

\subsection*{(3) \textmd{Generalizing $AA^T$ over $\F_2$}}

As a first step, we describe the entries in the matrix $AA^T \in M_{m \times m}(\F_2)$. Note that we are now considering $AA^T$ as a matrix over the field $\F_2$, as opposed to $\R$. Recall the definitions of addition and multiplication in $\F_2$ from the abstract.

Consider then the entries of $AA^T \in M_{m \times m}(\F_2)$:

$$
AA^T=
\begin{bmatrix}
\sum_{k=1}^{n} a_{1k}a_{1k} & ... & \sum_{k=1}^{n} a_{1k}a_{mk}  \\
\sum_{k=1}^{n} a_{2k}a_{1k}  & ... & \sum_{k=1}^{n} a_{2k}a_{mk}  \\
... & ... & ... \\
\sum_{k=1}^{n} a_{mk}a_{1k} &  ... & \sum_{k=1}^{n} a_{mk}a_{mk} 
\end{bmatrix}
$$

With the definition of addition and multiplication over $\F_2$, we find that $a^T_{ij}$ is 1 if the clubs $C_i, C_j$ share an odd number of members in common and 0 if the clubs $C_i, C_j$ share an even number of members in common.

\subsection*{(4) \textmd{Restricting $AA^T$ to our Problem}}

With our restrictions that we defined earlier, denoted (a) and (b), we know that when $i=j$ then $A^T_{ij} = 1$ since every club has an odd number of members (therefore, every club has an odd number of members in common with itself). We also know that when $i \ne j$, then $A^T_{ij} = 0$, since every club may only have an even number of members in common. Therefore, our matrix $AA^T \in M_{m \times m}(\F_2)$ is exactly:

\renewcommand{\arraystretch}{0.8}
$$
AA^T =
\begin{bmatrix}
1 & 0 & ... & 0  \\
0 & 1 & ... & 0  \\
0 & 0 & ... & 0  \\
0 & 0 & ... & 1 
\end{bmatrix}
$$

In other words, $AA^T = I_m$ where $I_m \in M_{m \times m}(\F_2)$ is the identity matrix with $m$ rows and $m$ columns over $\F_2$.

The rank of $AA^T$ is therefore exactly $m$, as the identity matrix $I_m$ has exactly $m$ linearly independent columns. This is the number of student clubs.

\subsection*{(5) \textmd{Proof on Limit on Student Clubs with Restrictions}}

Consider two arbitrary matrices $A,B$ over a field $\F$, where we can form $AB$. Note that this means that $A$ and $B$ must be the appropriate dimensions. We want to show that $ rank(AB) \le min\{rank(A), rank(B)\}$.

To do this, we must show that $rank(AB) \le rank(A)$ and $rank(AB) \le rank(B)$. \\

Show: $rank(AB) \le rank(A)$

Proof:

Let $v \in Im(AB)$. \\
$\exists u$ s.t. $v = ABu$ by definition. \\
Let $w = Bu$. So $v = Aw$. \\
Therefore, $v \in Im(A)$. \\
Conclude $Im(AB) \subseteq Im(A)$. \\
Therefore, $dim(Im(AB)) \le dim(Im(A))$. \\
Therefore, $rank(AB) \le rank(A)$. $\square$ \\

Show: $rank(AB) \le rank(B)$

Proof:

Let $\exists v \in ker(B)$. \\
Then $Bv = 0$. \\
Therefore, $ABv = A(Bv) = A(0) = 0$. \\
Conclude $v \in ker(AB)$. \\
So $ker(B) \subseteq ker(AB)$. \\
Therefore, $dim(ker(B)) \le dim(ker(AB))$. \\
Equivalently, $nullity(B) \le nullity(AB)$.
\vspace{0.5em}

Let $n$ be the dimension of the input space. \\
By rank-nullity, $rank(AB) + nullity(AB) = n$. \\
By previous proof, $nullity(B) \le nullity(AB)$. \\
Therefore, $rank(AB) + nullity(B) \le n$. \\
Equivalently, $rank(AB) \le n - nullity(B)$.
\vspace{0.5em}

Recall $AB$ and $B$ share the same input space. \\
By rank-nullity, $rank(B) + nullity(B) = n$ \\
Equivalently, $rank(B) = n - nullity(B)$.
\vspace{0.5em}

Therefore, we can say $rank(AB) \le rank(B)$. $\square$ \\

We conclude that $rank(AB) \le min\{rank(A),rank(B)\}$, as desired.

Finally, we deduce an upper bound for the number of clubs that $n$ students can participate in. We know that $AA^T = I_m \in M_{m \times m}(\F_2)$, where $m$ is the number of clubs. From the above result, we know that
$$rank(AA^T) \le min\{rank(A),rank(A^T)\}$$
Previously, we showed that $rank(AA^T) = m$. We also deduced that $rank(A) \le n$, which is true for any field $\F$. From this, we can say that
$$rank(AA^T) \le rank(A)$$
$$m \le n$$
so that the number of clubs must always be less than or equal to the number of students. With that, we conclude our discussion of a matrix representation for the combinatorics of student club participation. $\square$







\end{document}